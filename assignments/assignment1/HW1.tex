
\documentclass[12pt]{exam}
\printanswers
\usepackage{graphicx}
\usepackage{amsmath}
\usepackage{caption}
\usepackage{enumitem}
\usepackage{listings}
\usepackage{subcaption}
\usepackage[font={small,it}]{caption}

\graphicspath{ {images/} }
\title {High Performance scientific computing \\
Assignment 1} 
\author {Anshuman kumar \\
Roll no.  120010036}

\begin{document}
\maketitle


\begin{questions}
    \question The serial portion of an algorithm constitutes 20 \%.
    \begin{enumerate}[label=(\alph*)]
        \item What is the maximum achievable speedup for this algorithm? [5pts]
        \item If the desirable speedup is 50. What should be maximum 
            percentage of the serial portion for the algorithm? [5pts]
    \end{enumerate}
    \begin{solution} \\
        (a) Using Amdahl's law

        \[
            S(N) = \frac{1}{(1-P) + \frac{P}{N}}
        \]

        We got the maxinmum speed up when n = \(\infty\) \\ 
        We get Speed up = 1/0.2 = 5.0

        (b) The speed up needed is 50.0. Using the Amdahl's law we get 
        \[
            \frac{1}{(1-P)} = 50.0
        \]
        We get 1-P = 0.02  \\
        So finally we we should have maximum of 2\% Serial Code
    \end{solution}

    \question . Consider a hypothetical Von Neumann type machine shown in the
    figure. This machine has a cache of 1 KiloByte and a main memory of 10
    MegaBytes. The cost of fetching one floating point number (8 bytes in size)
    from the cache is 1 CPU cycle (or one clock tick). In the event the data
    needed is not in the cache, a chunk of data 1 KiloByte in size is fetched
    from the main memory to cache, replacing the contents of the cache.
    The cost of fetching data from main memory is 150 CPU cycles.

  Now, considering the following snippet code, where matrix A is initialised 
  to some integer values
  \begin{lstlisting}
  #define N 1023
  sum = 0;
  for( i=0; i<=N; i++ )
  for( j=0; j<=N; j++ )
  sum = sum + A[i][j];
  \end{lstlisting}
  \begin{enumerate}[label=(\alph*)]
      \item What is the cost in CPU cycles to fetch the requisite data for the matrix A to
  perform the computation (code is in C)? Disregard the cost for the variable sum
  [20pts]
  \item What will be the cost again in terms of CPU cycles, if the exact same code is
  written in FORTRAN? [20pts]
    \end{enumerate}
    \begin{solution} 
  \begin{enumerate}[label=(\alph*)]
      \item
        C is row-major order language. \\
        Size of int =  4bytes \\
        Max No of int in cache = 1024/4 = 256 \\
        \\ 
        For Internal loop: \\
        No of times cpu accesing main memory = 1024/256 = 4 \\ 
        No of times cpu accessing cache = 512  ( Taking into account it can transfer 8 bytes at one cpu cycle) \\
        Total no of CPU cycle = 512 + 150*4 =  1124 \\ \\

        No of external for loop = 1024 \\
        Total No of cpu cycles = 1024* 1124 = 1150976 \\

    \item
        Fortan is column major order language.So for taking a sum it has
        to access the memory again and again.Everytime it has to access main memory
        for getting a  number.\\ \\
        No of times cpu accesses main memory = 1024* 1024 =  1048576 \\
        No of times cpu accesses cache = 1024* 1024 =  1048576 \\
        Total Cpu cycle = 1048576*150 + 1048576 = 158334976

        
    \end{enumerate}
    \end{solution}

\end{questions}
\end{document}
